% quiver style
\usepackage{tikz-cd}

\tikzset{mid arrow/.style 2 args={postaction={decorate,decoration={
              markings,
              mark=at position {#1} with {\arrow[#2]{to}}
            }}}}

% Draw arc from circle centre
\def\centerarc[#1](#2)(#3:#4:#5)% Syntax: [draw options] (center) (initial angle:final angle:radius)
{ \draw[#1] ($(#2)+({#5*cos(#3)},{#5*sin(#3)})$) arc (#3:#4:#5); }

% TikZ arrowhead/tail styles.
\tikzset{tail reversed/.code={\pgfsetarrowsstart{tikzcd to}}}
\tikzset{2tail/.code={\pgfsetarrowsstart{Implies[reversed]}}}
\tikzset{2tail reversed/.code={\pgfsetarrowsstart{Implies}}}
% TikZ arrow styles.
\tikzset{no body/.style={/tikz/dash pattern=on 0 off 1mm}}

% useful macro for class
\usepackage{xfrac}
\usepackage{physics}

% \newcommand{\varsf}{\usefont{T1}{lmssq}{md}{n}aas}
\DeclareFontFamily{T1}{lmssq}{}
\DeclareFontShape{T1}{lmssq}{md}{n}{ <-> s*[0.87] cs-lmssq8 }{}
\DeclareMathAlphabet{\fundamental}{T1}{lmssq}{md}{n}

\NewDocumentCommand\set{sm}{%
  \IfBooleanTF#1%
  {\{#2\}}% If a star is seen
  {\qty{#2}}% If no star is seen
}

\NewDocumentCommand\inner{mm}{%
  \left\langle#1,#2\right\rangle%
}

\newcommand*{\euler}{\fundamental{e}}
\newcommand*{\im}{\fundamental{i}}
\newcommand*{\ihat}{\bm{\hat{\textbf{\i}}}}
\newcommand*{\jhat}{\bm{\hat{\textbf{\j}}}}
\newcommand*{\khat}{\bm{\hat{\textbf{k}}}}
\newcommand*{\basis}[1]{\mathbf{#1}}
\newcommand*{\mean}[1]{\langle #1 \rangle}
\newcommand*{\df}{\mathop{}\!\mathrm{d}}
\newcommand*{\N}{\ensuremath{\mathbb{N}}}
\newcommand*{\Z}{\ensuremath{\mathbb{Z}}} % Integers
\newcommand*{\R}{\ensuremath{\mathbb{R}}} % Real numbers
\newcommand*{\C}{\ensuremath{\mathbb{C}}} % Complex numbers
\newcommand*{\Q}{\ensuremath{\mathbb{Q}}} % Rational numbers

\DeclareMathOperator{\sgn}{sign}
\DeclareMathOperator{\spn}{span}
\DeclareMathOperator{\diag}{diag}
\DeclareMathOperator{\diam}{diam}
\DeclareMathOperator*{\argmax}{argmax}
\DeclareMathOperator*{\argmin}{argmin}

\newcommand{\res}{\Res}
\renewcommand{\leq}{\leqslant}
\renewcommand{\geq}{\geqslant}

\let\implies\Rightarrow
\let\impliedby\Leftarrow
\makeatletter
\renewcommand*\iff{%
  \relax\if@display
    \expandafter\Longleftrightarrow
  \else
    \expandafter\Leftrightarrow
  \fi
}
\makeatother
\newcommand{\uniformto}{\rightrightarrows}

\usepackage{bm}
\usepackage{bboldx}
% \usepackage[bb=mt]{mathalpha}

% Define \thinoverline for use in \conj
\usepackage[usestackEOL]{stackengine}
\usepackage{scalerel}
\def\thinoverline#1{\ThisStyle{%
    \setbox0=\hbox{$\SavedStyle#1$}%
    \stackengine{1.2\LMpt}{$\SavedStyle#1$}{\rule{\wd0}{.2\LMpt}}{O}{c}{F}{F}{S}%
  }}

% Define \conj
\ExplSyntaxOn
\NewDocumentCommand{\conj}{m}{
  \int_compare:nTF {\tl_count:n { #1 } > 1}
  {
    \thinoverline{#1}
  }
  {
    \mkern 2.5mu\thinoverline{\mkern-2.5mu#1\mkern-0.5mu}\mkern 0.5mu
  }
}

% Remove extra space below equations
\newenvironment{NoBelowDisplaySkip}{%
  \l_MT_above_intertext_sep=\belowdisplayskip
  \l_MT_above_shortintertext_sep=\belowdisplayshortskip
  \belowdisplayskip=0pt
  \belowdisplayshortskip=0pt
}{%
}
\ExplSyntaxOff

% Augmented matrix
\newenvironment{amatrix}[1]{%
  \left[\begin{array}{@{}*{#1}{c}|c@{}}
      }{%
    \end{array}\right]
}

% Block matrix
\newenvironment{blmatrix}[1]{%
  \left[\begin{array}{#1}
      }{%
    \end{array}\right]
}

% Optional parameter for matrix environment to stretch vertically
\makeatletter
\renewcommand*\env@matrix[1][\arraystretch]{%
  \edef\arraystretch{#1}%
  \hskip -\arraycolsep
  \let\@ifnextchar\new@ifnextchar
  \array{*\c@MaxMatrixCols c}}
\makeatother

% \delimgrowth=1