\section{Introduction}\label{sec:1}
\subsection{A subsection}

Let's showcase some of the environments.
\begin{definition}[title=Template, label=template]
	A document or file having a preset format, used as a starting point for a particular application so that the format does not have to be recreated each time it is used.
\end{definition}
\begin{definition*}
	This definition is not numbered and does not have a title.
\end{definition*}

We can also reference such environments in our \hyperref[def:template]{template}.

Theorems can be proved in the same box,

\begin{theorem}[label=addone]
	\begin{equation*}
		1 + 1 = 2
	\end{equation*}
	\begin{proof}
		Trivial
	\end{proof}
\end{theorem}

or in a separate box, as seen in \Cref{app:1}.

There are similar environments for the following:
\begin{itemize}
	\item propositions
	\item conjectures
	\item corollaries
	\item lemmas
\end{itemize}

Exercises and answers behave like theorems and proofs.
\begin{exercise}
	Show that
	\begin{equation*}
		\int_{0}^{2\pi}\;dx = 2\pi
	\end{equation*}
	\begin{answer}
		\begin{align*}
			\int_{0}^{2\pi}\;dx & = \Bigl[x\Bigr]_{0}^{2\pi} \\
			                    & = 2\pi - 0                 \\
			                    & = 2\pi
		\end{align*}
	\end{answer}
\end{exercise}

\begin{example*}
	Calculate \(\partial_{v}f(x,y)\) for the function \(f(x,y)=x^{2}-y^{2}\) in the direction of \(v=(a,b)\).
	\begin{answer}[Solution]
		\begin{align*}
			f\left((x,y)+t(a,b)\right)=(x+ta)^{2}-(y+tb)^{2}    & =x^{2}+2tax+t^{2}a^{2}+y^{2}-2tby-t^{2}b^{2} \\
			\therefore \frac{d}{dt}f(x+tv)                      & =2ax+2ta^{2}-2by-2tb^{2},                    \\
			\partial_{v}f(x,y)= \frac{d}{dt}f(x+tv)\bigg|_{t=0} & =2ax-2by.
		\end{align*}
	\end{answer}
	\begin{note}
		Environments can be nested.
	\end{note}
\end{example*}

\begin{notation}
	For \(1\leq i\leq n\), \(\partial_{v_{i}}f(x)\) is called the \emph{i\textsuperscript{th}-partial derivative} of \(f:U\rightarrow \mathbb{R}^{k}\) at \(x \in U\).
\end{notation}

\section{Other environments}

\subsection{Algorithms}

The following example is taken from the \texttt{algorithm2e} \href{http://mirror.ctan.org/macros/latex/contrib/algorithm2e/doc/algorithm2e.pdf}{package documentation}.

\begin{algorithm}[title=Example]
	\KwData{this text}
	\KwResult{how to write algorithm with \LaTeX2e}
	initialization\;
	\While{not at end of this document}{
		read current\;
		\eIf{understand}{
			go to next section\;
			current section becomes this one\;
		}{
			go back to the beginning of current section\;
		}
	}
\end{algorithm}

\subsection{Formulas}

Useful for physics notes.

\begin{formula}[title=Gibb's entropy]
	\begin{equation}
		S \coloneq -k_{\mathrm{B}}\sum_{i}P_{i}\ln P_{i}
	\end{equation}
\end{formula}
